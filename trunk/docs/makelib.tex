\documentclass{article}
\input macros

\def\makelib{\textbf{makelib} }

\begin{document}

\SH{Name}
\makelib --- create, manipulate. and query OMF library files.

\SH{Synopsis}

\makelib [options] \textit{library file} [\textit{object files \dots}]

\SH{Description}

\makelib is a utility for creating and manipulating OMF library files.  \makelib can add, delete, and extract files from the library, as well as list the library contents.

An OMF library file 
(filetype \code{\$B2}, \code{LIB}) is a collection of OMF object files and a dictionary
header.

The dictionary header is stored in an OMF segment, kind \code{\$08}, Library Dictionary Segment.  The segment contains 3 \code{LCONST} records, described below.  All numbers are stored in little-endian format.  All strings are variable length, preceeded by a length byte.

The first contains filename information.

\par

\begin{tabular}{p{8em} p{6em}  p{.5\textwidth}}
\code{File Number} & word & Unique number for this file. Non-zero value.\\
\code{File Name} & string & The name of the original OMF file.\\
\end{tabular}

The second contains symbol information.

\begin{tabular}{p{8em} p{6em}  p{.5\textwidth}}
\code{Name Offset} & long word & offset into the symbol name table. \\
\code{File Number} & word & File in which this symbol was originally defined.\\
\code{Private Flag} & word & Zero if this symbol is public, non-zero if this symbol is private. \\
\code{Segment Offset} & long word & Absolute offset into the file to access the segment in which this symbol is defined.\\
\end{tabular}

The third contains symbol names and is sorted alphabetically.

\begin{tabular}{p{8em} p{6em}  p{.5\textwidth}}
\code{Symbol Name} & string & Name of the symbol \\
\end{tabular}


\SH{Options}

\makelib recognizes the following options:
\begin{optionlist}
	\item [-a]	
	Add files to the library, creating the library if necessary.  The files to be added should be
	OMF object files (\textit{i.e.}, not linked).
	\item [-d]
	Delete the specified files from the library.
	\item [-x]
	Extract the specified files from the library.  The files will not deleted from the library
	unless the -d flag is also specified.
	\item [-t] 
	Print the dictionary contents.  
	\item [-v]
	Be more verbose.
	\item [-h]
	Print usage and version information.
\end{optionlist}


\end{document}