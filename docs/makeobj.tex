\documentclass{article}
\input macros

\def\makeobj{\textbf{makeobj} }

\begin{document}

\SH{Name}
\makeobj --- convert a binary file to an OMF file.

\SH{Synopsis}

\makeobj [options] \textit{file}

\SH{Description}

\makeobj converts a binary file (such as image or sound data) to an OMF (object module format) file.  The OMF file can be linked and accessed as if it was externally declared.

\SH{Example}

\begin{verbatim}
makeobj -o binary.o -n picture binary.data
\end{verbatim}

The data can now be accessed as if it were an external array:

\begin{verbatim}
extern char picture[];
\end{verbatim}


\SH{Options}

\makeobj recognizes the following options:
\begin{optionlist}
	\item [-a number] Set the OMF file segment.  The alignment must be a power of 2.  The 
	default alignment is 0.
	\item [-n name]
	Set the segment name. The default loadname is the same as the input filename 
	(minus any file extension). This is the name of the data in your program. C programs
	are case sensitive. If you're using assembly language, the segment name should be
	in capital letters if case sensitivity is off. 
	\item [-l name]
	Set the segment load name.  The default load name is blank.  The segment load name is used
	for splitting large programs into multiple segments. 
	\item [-k kind]
	Set the segment kind.  Valid values are ``CODE'', ``DATA'', ``INIT'', or ``STACK''.  
	The default kind is DATA.  Please note that DATA segments may cross bank boundaries when
	loaded.  If your data is < 65,535 (\texttt{\$ffff}) bytes in length and you do not want it to cross
	bank boundaries, specify the CODE kind. 
	\item [-o file]
	Set the output file name.  The default output file name is the same as the input filename, 
	but with a ``.o'' extension.
	\item [-h]
	Display help and version information.
	
\end{optionlist}

\end{document}