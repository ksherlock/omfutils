\documentclass{article}
\input macros

\def\dumpobj{\textbf{dumpobj} }

\begin{document}

\SH{Name}
\dumpobj --- display OMF file contents.

\SH{Synopsis}
\dumpobj [options] \it{omf-file} [\it{segment\_name \dots}]

\SH{Description}

\dumpobj dumps header and opcode information from an OMF file.  If segment names are specified, 
only those segments will be dumped.  If no segment names are specified, all segments will be dumped.

\SH{Options}

\dumpobj recognizes the following options:
\begin{optionlist}
	\item [-d]
	Disassemble CODE and INIT segments.  If specified twice (\it{i.e.} -dd), all segments will
	be disassembled.
	\item [-h]
	Display usage and version information.
	\item[-D]
	Do not do a hexdump of CONST, LCONST, or SUPER record data.
	\item[-H]
	Only the header information will be displayed.
	
\end{optionlist}

\SH{Files}

When disassembling data, \dumpobj will attempt to interpret toolbox and GS/OS calls.  To do so, it attempts to open files called ``tools.txt'' and ``gsos.txt'' located in /etc, /usr/etc, /usr/local/etc/, or the current directory.

\end{document}
